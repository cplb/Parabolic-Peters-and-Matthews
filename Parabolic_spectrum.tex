% Group addresses by affiliation; use superscriptaddress for long
% author lists, or if there are many overlapping affiliations.
% For Phys. Rev. appearance, change preprint to twocolumn.
% Choose pra, prb, prc, prd, pre, prl, prstab, prstper, or rmp for journal
%  Add 'draft' option to mark overfull boxes with black boxes
%  Add 'showpacs' option to make PACS codes appear
%  Add 'showkeys' option to make keywords appear
\documentclass[aps,prd,reprint,groupedaddress]{revtex4-1}


\begin{document}

%\preprint{}

%Title of paper
\title{The Parabolic Limit for Peters and Matthews Energy Spectra}

% \affiliation command applies to all authors since the last
% \affiliation command. The \affiliation command should follow the
% other information
% \affiliation can be followed by \email, \homepage, \thanks as well.
\author{Jonathan R. Gair}
\email[]{jgair@ast.cam.ac.uk}
%\affiliation{Institute of Astronomy, Madingley Road, Cambridge, CB3 0HA, United Kingdom}
\author{Christopher P. L. Berry}
\email[]{cplb2@ast.cam.ac.uk}
\affiliation{Institute of Astronomy, Madingley Road, Cambridge, CB3 0HA, United Kingdom}

\date{\today}

\begin{abstract}
% insert abstract here
\end{abstract}

% insert suggested PACS numbers in braces on next line
\pacs{}

%\maketitle must follow title, authors, abstract, \pacs, and \keywords
\maketitle

\section{}

Peters and Matthews\cite{Peters1963} give the power radiated into the $n$th harmonic of the orbital angular frequency as
\begin{equation}
P(n) = \frac{32}{5}\frac{G^4}{c^5}\frac{M_\bullet^2\mu^2(M_\bullet + \mu)(1-e)^5}{r\sub{p}^5}g(n,e)
\label{eq:PM_P}
\end{equation}
where the function $g(n,e)$ is defined in terms of Bessel functions of the first kind
\begin{align}
g(n,e) = {} & \frac{n^4}{32}\left\{\left[J_{n-2}(ne) - 2eJ_{n-1}(ne) + \frac{2}{n}J_n(ne) + 2eJ_{n+1}(ne) - J_{n+2}(ne)\right]^2 \right. \nonumber \\
 & + \left. \left(1 - e^2\right)\left[J_{n-2}(ne) - 2J_n(ne) + J_{n+2}(ne)\right]^2 + \frac{4}{3n^2}\left[J_n(ne)\right]^2\right\}.
\end{align}
The Keplerian orbital frequency is
\begin{align}
{\omega_0}^2 = {} & \frac{G(M_\bullet + \mu)(1-e)^3}{r\sub{p}^3}\\
 = {} & (1-e)^3{\omega\sub{c}}^2,
\label{eq:Kepler_freq}
\end{align}
where $\omega\sub{c}$ is defined as the orbital angular frequency of a circular orbit of radius equal to $r\sub{p}$. The total energy radiated into the $n$th harmonic, that is at frequency $\omega_n = n\omega_0$, is the power multiplied by the orbital period
\begin{equation}
E(n) = \frac{2\pi}{\omega_0}P(n);
\label{eq:E(n)}
\end{equation}
as $e \rightarrow 1$ for a parabolic orbit, $\omega_0 \rightarrow 0$ so the orbital period becomes infinite. We may therefore identify the energy radiated per orbit with the total orbital energy radiated. Since the spacing of harmonics is $\Delta\omega = \omega_0$, we may identify the energy spectrum
\begin{equation}
\left.\diff{E}{\omega}\right|_{\omega_n}\omega_0 = E(n).
\end{equation}
Using the above relations, and changing to linear frequency $2\pi f = \omega$, we obtain
\begin{align}
\left.\diff{E}{f}\right|_{f_n} = {} & \frac{128\pi^2}{5}\frac{G^3}{c^5}\frac{M_\bullet^2\mu^2}{r\sub{p}^2}(1-e)^2g(n,e) \\
 = {} & \frac{4\pi^2}{5}\frac{G^3}{c^5}\frac{M_\bullet^2\mu^2}{r\sub{p}^2}\ell(n,e),
\label{eq:PM_spectrum}
\end{align}
where we have defined the function $\ell(n,e)$ in the last line. For a parabolic orbit, we now have to take the limit of $\ell(n,e)$ as $e \rightarrow 1$. For this we shall use a number of properties of Bessel functions, and will make frequent reference to Watson\cite{Watson1995}.

We shall simplify $\ell(n,e)$ using the recurrence formulae (Watson\cite{Watson1995} 2.12)
\begin{align}
J_{\nu-1}(z) + J_{\nu+1}(z) = {} & \frac{2\nu}{z}J_\nu(z)\\
J_{\nu-1}(z) - J_{\nu+1}(z) = {} & 2J'_\nu(z).
\end{align}
We shall also eliminate $n$ using
\begin{align}
n = {} & \frac{\omega_n}{\omega_0} \nonumber \\
= {} & (1-e)^{-3/2}\widetilde{f}.
\end{align}
where $\widetilde{f} = \omega_n/\omega\sub{c} = f_n/f\sub{c}$ is a dimensionless frequency. We begin by breaking $\ell$ into three parts
\begin{align}
\ell = {} & \underbrace{(1-e)^2n^4\left[J_{n-2} - 2eJ_{n-1} + \frac{2}{n}J_n + 2eJ_{n+1} - J_{n+2}\right]^2}_{\ell_1} \nonumber \\
  & + \underbrace{(1-e)^3(1+e)n^4\left[J_{n-2} - 2J_n + J_{n+2}\right]^2}_{\ell_2} + \underbrace{\frac{4(1-e)^2n^2}{3}\left[J_n\right]^2}_{\ell_3}.
\end{align}
We have suppressed the argument of the Bessel functions for brevity. Tackling each term of $\ell$ in turn we obtain
\begin{align}
\ell_1(\widetilde{f},e) = {} & \left[\frac{4(1+e)\widetilde{f}^2}{e}\frac{J'_n}{1-e} + 2\frac{e-2}{e}\widetilde{f}\frac{J_n}{(1-e)^{1/2}}\right]^2\\
\ell_2(\widetilde{f},e) = {} & 16(1+e)\left[\frac{(1+e)\widetilde{f}^2}{e^2}\frac{J_n}{(1-e)^{1/2}} - \widetilde{f}\frac{J'_n}{e}\right]^2\\
\ell_3(\widetilde{f},e) = {} & \frac{4\widetilde{f}^2}{3}\left[{J_n}{(1-e)^{1/2}}\right]^2.
\end{align}
To take the limit of these we need to find the limiting behaviour of Bessel functions. We shall define two new functions
\begin{equation}
A(\widetilde{f}) = \lim_{e\rightarrow 1}\left\{\frac{J_n}{(1-e)^{1/2}}\right\}; \quad B(\widetilde{f}) = \lim_{e\rightarrow 1}\left\{\frac{J'_n}{1-e}\right\}.
\end{equation}
To give a well defined energy spectrum, both of these must be finite. In this case we see that the second term in $\ell_2$ should go to zero.

The Bessel function has an integral representation
\begin{equation}
J_\nu(z) = \recip{\pi}\intd{0}{\pi}{\cos(\nu\theta - z\sin\theta)}{\theta},
\end{equation}
we want the limit of this for $\nu \rightarrow \infty$, $z \rightarrow \infty$, with $z \leq \nu$. We will use the stationary phase approximation to argue that the predominant contribution to the integral comes from when the argument of the cosine is approximately zero, that is for small $\theta$ (Watson\cite{Watson1995} 8.2, 8.43). In this case we have
\begin{align}
J_\nu(z) \sim {} & \recip{\pi}\intd{0}{\pi}{\cos\left(\nu\theta - z\theta + \frac{z}{6}\theta^3\right)}{\theta}\\
 \sim {} & \recip{\pi}\intd{0}{\infty}{\cos\left(\nu\theta - z\theta + \frac{z}{6}\theta^3\right)}{\theta};
\end{align}
this last expression is an Airy integral. The Airy integral has a standard form (Watson\cite{Watson1995} 6.4)
\begin{equation}
\intd{0}{\infty}{\cos(t^3 + xt)}{t} = \frac{\sqrt{x}}{3}K_{1/3}\left(\frac{2x^{3/2}}{3^{3/2}}\right),
\end{equation}
where $K_\nu(z)$ is a modified Bessel function of the second kind. Using this to evaluate our limit gives
\begin{equation}
J_\nu(z) \sim \recip{\pi}\sqrt{\frac{2(\nu - z)}{3z}}K_{1/3}\left(\frac{2^{3/2}}{3}\sqrt{\frac{(\nu -z)^3}{z}}\right).
\end{equation}
For our particular case we have
\begin{equation}
\nu = (1 - e)^{-3/2}\widetilde{f}; \quad z = (1 - e)^{-3/2}e\widetilde{f};
\end{equation}
\begin{equation}
\frac{\nu - z}{z} = (1 - e); \quad \frac{(\nu - z)^3}{z} = \widetilde{f}^2;
\end{equation}
so we find
\begin{equation}
J_n(ne) \sim \recip{\pi}\sqrt{\frac{2}{3}}(1-e)^{1/2}K_{1/3}\left(\frac{2^{3/2}\widetilde{f}}{3}\right),
\end{equation}
thus
\begin{equation}
A(\widetilde{f}) = \recip{\pi}\sqrt{\frac{2}{3}}K_{1/3}\left(\frac{2^{3/2}\widetilde{f}}{3}\right)
\end{equation}
is well defined.

Now finding the derivative
\begin{align}
J'_\nu(z) = {} & \recip{2}\left[J_{\nu-1}(z) - J_{\nu+1}(z)\right] \nonumber \\
 \sim {} & \recip{2\pi}\left[\sqrt{\frac{2(\nu -1 - z)}{3z}}K_{1/3}\left(\frac{2^{3/2}}{3}\sqrt{\frac{(\nu - 1 - z)^3}{z}}\right) \right. \nonumber \\
  & \left. - \sqrt{\frac{2(\nu +1 - z)}{3z}}K_{1/3}\left(\frac{2^{3/2}}{3}\sqrt{\frac{(\nu + 1 - z)^3}{z}}\right)\right].
\end{align}
For our case
\begin{align}
\sqrt{\frac{\nu \pm 1 - z}{z}} = {} & (1 - e)^{1/2}\left[1 \pm \frac{(1-e)^{1/2}}{2\widetilde{f}} + \ldots \, \right];\\
\sqrt{\frac{(\nu \pm 1 - z)^{3/2}}{z}} = {} & \widetilde{f}\left[1 \pm \frac{3(1-e)^{1/2}}{2\widetilde{f}} + \ldots \, \right];
\end{align}
and so
\begin{align}
J'_n(ne) \sim {} & \recip{2\pi}\sqrt{\frac{2}{3}}(1-e)^{1/2}\left\{\left[1 - \frac{(1-e)^{1/2}}{2\widetilde{f}}\right]K_{1/3}\left(\frac{2^{3/2}\widetilde{f}}{3}\left[1 - \frac{3(1-e)^{1/2}}{2\widetilde{f}}\right]\right) \right. \nonumber \\
 & \left. - \left[1 + \frac{(1-e)^{1/2}}{2\widetilde{f}}\right]K_{1/3}\left(\frac{2^{3/2}\widetilde{f}}{3}\left[1 - \frac{3(1-e)^{1/2}}{2\widetilde{f}}\right]\right)\right\}\nonumber \\
 \sim {} & \frac{-1}{2\pi}\sqrt{\frac{2}{3}}(1-e)\left[2^{3/2}K'_{1/3}\left(\frac{2^{3/2}\widetilde{f}}{3}\right) + \recip{\widetilde{f}}K_{1/3}\left(\frac{2^{3/2}\widetilde{f}}{3}\right)\right].
\end{align}
We may re-express the derivative using the recurrence formula (Watson\cite{Watson1995} 3.71)
\begin{equation}
K_{\nu-1}(z) - K_{\nu+1}(z) = -2K'_\nu(z)
\end{equation}
to give
\begin{equation}
J'_n(ne) \sim \frac{1-e}{\sqrt{3}\pi}\left[K_{-2/3}\left(\frac{2^{3/2}\widetilde{f}}{3}\right) + K_{4/3}\left(\frac{2^{3/2}\widetilde{f}}{3}\right) - \recip{\sqrt{2}\widetilde{f}}K_{1/3}\left(\frac{2^{3/2}\widetilde{f}}{3}\right)\right].
\end{equation}
And so finally,
\begin{equation}
B(\widetilde{f}) = \recip{\sqrt{3}\pi}\left[K_{-2/3}\left(\frac{2^{3/2}\widetilde{f}}{3}\right) + K_{4/3}\left(\frac{2^{3/2}\widetilde{f}}{3}\right) - \recip{\sqrt{2}\widetilde{f}}K_{1/3}\left(\frac{2^{3/2}\widetilde{f}}{3}\right)\right],
\end{equation}
which is also well defined.

Having obtained expressions for $A(\widetilde{f})$ and $B(\widetilde{f})$ in terms of standard functions, we may now calculate the energy spectrum for a parabolic orbit. From \eqnref{PM_spectrum} we have
\begin{equation}
\diff{E}{f} = \frac{4\pi^2}{5}\frac{G^3}{c^5}\frac{M_\bullet^2\mu^2}{r\sub{p}^2}\ell\left(\frac{f}{f\sub{c}}\right),
\label{eq:PM_dEdf}
\end{equation}
where we have used the limit
\begin{align}
\ell(\widetilde{f}) = {} & \lim_{e \rightarrow 1}\left\{\ell(n,e)\right\} \nonumber \\
 = {} & \left[8\widetilde{f}B(\widetilde{f}) - 2\widetilde{f}A(\widetilde{f})\right]^2 + \left(128\widetilde{f}^4 + \frac{4\widetilde{f}^2}{3}\right)\left[A(\widetilde{f})\right]^2.
\end{align}

To check the validity of this limit we may calculate the total energy radiated. We should be able to calculate this by integrating \eqnref{PM_dEdf} over all frequencies, or alternatively by summing the energy radiated into each harmonic. For consistency, the two approaches should yield the same result. First, summing over harmonics we obtain
\begin{align}
E\sub{sum} = {} & \sum_n E(n) \nonumber \\
 = {} & \frac{64\pi}{5}\frac{G^3}{c^5}\frac{M_\bullet^2\mu^2}{r\sub{p}^2}\omega\sub{c}(1-e)^{7/2}\sum_n g(n,e),
\end{align}
where we have used equations \eqref{eq:PM_P}, \eqref{eq:Kepler_freq} and \eqref{eq:E(n)}. Peters and Matthews\cite{Peters1963} provide the result
\begin{equation}
\sum_n g(n,e) = \frac{1 + \nicefrac{73}{24}\, e^2 + \nicefrac{37}{96}\, e^4}{(1-e^2)^{7/2}}.
\end{equation}
Using this,
\begin{equation}
E\sub{sum} = \frac{64\pi}{5}\frac{G^3}{c^5}\frac{M_\bullet^2\mu^2}{r\sub{p}^2}\omega\sub{c}\frac{1 + \nicefrac{73}{24}\, e^2 + \nicefrac{37}{96}\, e^4}{(1+e^2)^{7/2}}.
\end{equation}
This is perfectly well behaved as $e \rightarrow 1$. Taking the limit for a parabolic orbit, the total energy radiated is
\begin{equation}
E\sub{sum} = \frac{85\pi}{2^{5/2}3}\frac{G^3}{c^5}\frac{M_\bullet^2\mu^2}{r\sub{p}^2}\omega\sub{c}.
\end{equation}
Integrating over the energy spectrum, \eqnref{PM_dEdf}, gives
\begin{align}
E\sub{int} = {} & \intd{0}{\infty}{\diff{E}{f}}{f} \nonumber \\
 = {} & \frac{2\pi}{5}\frac{G^3}{c^5}\frac{M_\bullet^2\mu^2}{r\sub{p}^2}\omega\sub{c}\intd{0}{\infty}{\ell(\widetilde{f})}{\widetilde{f}}.
\end{align}
The integral can be easily evaluated numerically showing
\begin{align}
\intd{0}{\infty}{\ell(\widetilde{f})}{\widetilde{f}} = {} & 12.5216858\ldots \nonumber \\
 = {} & \frac{425}{2^{7/2}3},
\end{align}
and so we find that the two total energies are consistent
\begin{align}
\label{eq:PM_total}
E\sub{int} = {} & \frac{85\pi}{2^{5/2}3}\frac{G^3}{c^5}\frac{M_\bullet^2\mu^2}{r\sub{p}^2}\omega\sub{c} \\
 = {} & E\sub{sum}.
\end{align}

\subsection{}
\subsubsection{}

% If in two-column mode, this environment will change to single-column
% format so that long equations can be displayed. Use
% sparingly.
%\begin{widetext}
% put long equation here
%\end{widetext}

% figures should be put into the text as floats.
% Use the graphics or graphicx packages (distributed with LaTeX2e)
% and the \includegraphics macro defined in those packages.
% See the LaTeX Graphics Companion by Michel Goosens, Sebastian Rahtz,
% and Frank Mittelbach for instance.
%
% Here is an example of the general form of a figure:
% Fill in the caption in the braces of the \caption{} command. Put the label
% that you will use with \ref{} command in the braces of the \label{} command.
% Use the figure* environment if the figure should span across the
% entire page. There is no need to do explicit centering.

% \begin{figure}
% \includegraphics{}%
% \caption{\label{}}
% \end{figure}

% Surround figure environment with turnpage environment for landscape
% figure
% \begin{turnpage}
% \begin{figure}
% \includegraphics{}%
% \caption{\label{}}
% \end{figure}
% \end{turnpage}

% tables should appear as floats within the text
%
% Here is an example of the general form of a table:
% Fill in the caption in the braces of the \caption{} command. Put the label
% that you will use with \ref{} command in the braces of the \label{} command.
% Insert the column specifiers (l, r, c, d, etc.) in the empty braces of the
% \begin{tabular}{} command.
% The ruledtabular enviroment adds doubled rules to table and sets a
% reasonable default table settings.
% Use the table* environment to get a full-width table in two-column
% Add \usepackage{longtable} and the longtable (or longtable*}
% environment for nicely formatted long tables. Or use the the [H]
% placement option to break a long table (with less control than 
% in longtable).
% \begin{table}%[H] add [H] placement to break table across pages
% \caption{\label{}}
% \begin{ruledtabular}
% \begin{tabular}{}
% Lines of table here ending with \\
% \end{tabular}
% \end{ruledtabular}
% \end{table}

% Surround table environment with turnpage environment for landscape
% table
% \begin{turnpage}
% \begin{table}
% \caption{\label{}}
% \begin{ruledtabular}
% \begin{tabular}{}
% \end{tabular}
% \end{ruledtabular}
% \end{table}
% \end{turnpage}

% Specify following sections are appendices. Use \appendix* if there
% only one appendix.
%\appendix
%\section{}

% If you have acknowledgments, this puts in the proper section head.
\begin{acknowledgments}
% put your acknowledgments here.
\end{acknowledgments}

% Create the reference section using BibTeX:
\bibliography{basename of .bib file}

\end{document}
%
% ****** End of file apstemplate.tex ******

