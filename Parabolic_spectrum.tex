% Group addresses by affiliation; use superscriptaddress for long
% author lists, or if there are many overlapping affiliations.
% For Phys. Rev. appearance, change preprint to twocolumn.
% Choose pra, prb, prc, prd, pre, prl, prstab, prstper, or rmp for journal
%  Add 'draft' option to mark overfull boxes with black boxes
%  Add 'showpacs' option to make PACS codes appear
%  Add 'showkeys' option to make keywords appear
\documentclass[aps,prd,reprint,groupedaddress]{revtex4-1}
\usepackage{slashed}
\usepackage[load=prefixed, load=abbr, alsoload=hep, alsoload=astro]{siunitx}

\newcommand{\eqnref}[1]{(\ref{eq:#1})}
\newcommand{\figref}[1]{fig.~\ref{fig:#1}}
\newcommand{\Figref}[1]{Fig.~\ref{fig:#1}}
\newcommand{\secref}[1]{section~\ref{sec:#1}}

\newcommand{\sub}[1]{\ensuremath{_\text{#1}}}
\newcommand{\super}[1]{\ensuremath{^\text{#1}}}
\newcommand{\dd}{\ensuremath{\mathrm{d}}}
\newcommand{\diff}[2]{\ensuremath{\frac{\dd {#1}}{\dd {#2}}}}
\newcommand{\intd}[4]{\ensuremath{\int_{#1}^{#2}{#3}\,\dd{#4}}}
\newcommand{\recip}[1]{\ensuremath{\frac{1}{#1}}}

\newcommand{\Ibar}{{\declareslashed{}{\text{-}}{0.04}{-0.2}{I}\slashed{I}}}

\begin{document}

%\preprint{}

%Title of paper
\title{The Parabolic Limit for Peters and Matthews Energy Spectra}

% \affiliation command applies to all authors since the last
% \affiliation command. The \affiliation command should follow the
% other information
% \affiliation can be followed by \email, \homepage, \thanks as well.
\author{Jonathan R. Gair}
\email[]{jgair@ast.cam.ac.uk}
%\affiliation{Institute of Astronomy, Madingley Road, Cambridge, CB3 0HA, United Kingdom}
\author{Christopher P. L. Berry}
\email[]{cplb2@ast.cam.ac.uk}
\affiliation{Institute of Astronomy, Madingley Road, Cambridge, CB3 0HA, United Kingdom}

\date{\today}

\begin{abstract}
% insert abstract here
\end{abstract}

% insert suggested PACS numbers in braces on next line
\pacs{}

%\maketitle must follow title, authors, abstract, \pacs, and \keywords
\maketitle

\section{Introduction}



In \secref{limit} we derive the parabolic limit for the Peters and Matthews energy spectrum. It is hoped that this analytic result can be used as a quick-and-easy method of investigating the energy flux from high eccentricity orbits, and as a means of checking that numerical codes give sensible results. The range of applicability of this result for equatorial orbits is discussed in \secref{application}.

\section{Parabolic Limit\label{sec:limit}}

\subsection{Energy Spectrum}

Peters and Matthews\cite{Peters1963} give the power radiated into the $n$th harmonic of the orbital angular frequency as
\begin{equation}
P(n) = \frac{32}{5}\frac{G^4}{c^5}\frac{M_1^2M_2^2(M_1 + M_2)(1-e)^5}{r\sub{p}^5}g(n,e)
\label{eq:PM_P}
\end{equation}
where the function $g(n,e)$ is defined in terms of Bessel functions of the first kind
\begin{align}
g(n,e) = {} & \frac{n^4}{32}\left\{\left[J_{n-2}(ne) - 2eJ_{n-1}(ne) + \frac{2}{n}J_n(ne) + 2eJ_{n+1}(ne) - J_{n+2}(ne)\right]^2 \right. \nonumber \\
 & + \left. \left(1 - e^2\right)\left[J_{n-2}(ne) - 2J_n(ne) + J_{n+2}(ne)\right]^2 + \frac{4}{3n^2}\left[J_n(ne)\right]^2\right\}.
\end{align}
The Keplerian orbital frequency is
\begin{align}
{\omega_0}^2 = {} & \frac{G(M_1 + M_2)(1-e)^3}{r\sub{p}^3}\\
 = {} & (1-e)^3{\omega\sub{c}}^2,
\label{eq:Kepler_freq}
\end{align}
where $\omega\sub{c}$ is defined as the orbital angular frequency of a circular orbit of radius equal to $r\sub{p}$. The energy radiated into the $n$th harmonic, that is at frequency $\omega_n = n\omega_0$, per orbit is the power multiplied by the orbital period
\begin{equation}
E(n) = \frac{2\pi}{\omega_0}P(n);
\label{eq:E(n)}
\end{equation}
as $e \rightarrow 1$ for a parabolic orbit, $\omega_0 \rightarrow 0$ and the orbital period becomes infinite. The energy radiated per orbit is the total energy radiated. Since the spacing of harmonics is $\Delta\omega = \omega_0$, we may identify the energy spectrum
\begin{equation}
\left.\diff{E}{\omega}\right|_{\omega_n}\omega_0 = E(n).
\end{equation}
Using the above relations, and changing to linear frequency $2\pi f = \omega$, we obtain
\begin{align}
\left.\diff{E}{f}\right|_{f_n} = {} & \frac{128\pi^2}{5}\frac{G^3}{c^5}\frac{M_1^2M_2^2}{r\sub{p}^2}(1-e)^2g(n,e) \\
 = {} & \frac{4\pi^2}{5}\frac{G^3}{c^5}\frac{M_1^2M_2^2}{r\sub{p}^2}\ell(n,e),
\label{eq:PM_spectrum}
\end{align}
where we have defined the function $\ell(n,e)$ in the last line. For a parabolic orbit, we now have to take the limit of $\ell(n,e)$ as $e \rightarrow 1$. For this we shall use a number of properties of Bessel functions, and will make frequent reference to Watson\cite{Watson1995}.

We simplify $\ell(n,e)$ using the recurrence formulae (Watson\cite{Watson1995} 2.12)
\begin{align}
J_{\nu-1}(z) + J_{\nu+1}(z) = {} & \frac{2\nu}{z}J_\nu(z)\\
J_{\nu-1}(z) - J_{\nu+1}(z) = {} & 2J'_\nu(z),
\end{align}
and eliminate $n$ using
\begin{align}
n = {} & \frac{\omega_n}{\omega_0} \nonumber \\
= {} & (1-e)^{-3/2}\widetilde{f},
\end{align}
where $\widetilde{f} = \omega_n/\omega\sub{c} = f_n/f\sub{c}$ is a dimensionless frequency. We begin by breaking $\ell$ into three parts
\begin{align}
\ell = {} & \underbrace{(1-e)^2n^4\left[J_{n-2} - 2eJ_{n-1} + \frac{2}{n}J_n + 2eJ_{n+1} - J_{n+2}\right]^2}_{\ell_1} \nonumber \\
  & + \underbrace{(1-e)^3(1+e)n^4\left[J_{n-2} - 2J_n + J_{n+2}\right]^2}_{\ell_2} + \underbrace{\frac{4(1-e)^2n^2}{3}\left[J_n\right]^2}_{\ell_3}.
\end{align}
We suppresse the argument of the Bessel functions for brevity. Tackling each term of $\ell$ in turn we obtain
\begin{align}
\ell_1(\widetilde{f},e) = {} & \left[\frac{4(1+e)\widetilde{f}^2}{e}\frac{J'_n}{1-e} + 2\frac{e-2}{e}\widetilde{f}\frac{J_n}{(1-e)^{1/2}}\right]^2;\\
\ell_2(\widetilde{f},e) = {} & 16(1+e)\left[\frac{(1+e)\widetilde{f}^2}{e^2}\frac{J_n}{(1-e)^{1/2}} - \widetilde{f}\frac{J'_n}{e}\right]^2;\\
\ell_3(\widetilde{f},e) = {} & \frac{4\widetilde{f}^2}{3}\left[{J_n}{(1-e)^{1/2}}\right]^2.
\end{align}
To find the limit of we define two new functions
\begin{equation}
A(\widetilde{f}) = \lim_{e\rightarrow 1}\left\{\frac{J_n(ne)}{(1-e)^{1/2}}\right\}; \qquad B(\widetilde{f}) = \lim_{e\rightarrow 1}\left\{\frac{J'_n(ne)}{1-e}\right\}.
\end{equation}
To give a well defined energy spectrum, both of these must be finite. In this case we see that the second term in $\ell_2$ should go to zero.

The Bessel function has an integral representation
\begin{equation}
J_\nu(z) = \recip{\pi}\intd{0}{\pi}{\cos(\nu\theta - z\sin\theta)}{\theta},
\end{equation}
we want the limit of this for $\nu \rightarrow \infty$, $z \rightarrow \infty$, with $z \leq \nu$. We will use the stationary phase approximation to argue that the dominant contribution to the integral comes from when the argument of the cosine is approximately zero, that is for small $\theta$ (Watson\cite{Watson1995} 8.2, 8.43). In this case we have
\begin{align}
J_\nu(z) \sim {} & \recip{\pi}\intd{0}{\pi}{\cos\left(\nu\theta - z\theta + \frac{z}{6}\theta^3\right)}{\theta}\\
 \sim {} & \recip{\pi}\intd{0}{\infty}{\cos\left(\nu\theta - z\theta + \frac{z}{6}\theta^3\right)}{\theta};
\end{align}
this last expression is an Airy integral. This has a standard form (Watson\cite{Watson1995} 6.4)
\begin{equation}
\intd{0}{\infty}{\cos(t^3 + xt)}{t} = \frac{\sqrt{x}}{3}K_{1/3}\left(\frac{2x^{3/2}}{3^{3/2}}\right),
\end{equation}
where $K_\nu(z)$ is a modified Bessel function of the second kind. Using this to evaluate our limit gives
\begin{equation}
J_\nu(z) \sim \recip{\pi}\sqrt{\frac{2(\nu - z)}{3z}}K_{1/3}\left(\frac{2^{3/2}}{3}\sqrt{\frac{(\nu -z)^3}{z}}\right).
\end{equation}
For our case
\begin{equation}
\nu = (1 - e)^{-3/2}\widetilde{f}; \qquad z = (1 - e)^{-3/2}e\widetilde{f};
\end{equation}
\begin{equation}
\frac{\nu - z}{z} = (1 - e); \qquad \frac{(\nu - z)^3}{z} = \widetilde{f}^2;
\end{equation}
thus
\begin{array}
J_n(ne) \sim \recip{\pi}\sqrt{\frac{2}{3}}(1-e)^{1/2}K_{1/3}\left(\frac{2^{3/2}\widetilde{f}}{3}\right),
\end{equation}
and the limiting function is well defined,
\begin{equation}
A(\widetilde{f}) = \recip{\pi}\sqrt{\frac{2}{3}}K_{1/3}\left(\frac{2^{3/2}\widetilde{f}}{3}\right).
\end{equation}

Finding the derivative
\begin{align}
J'_\nu(z) = {} & \recip{2}\left[J_{\nu-1}(z) - J_{\nu+1}(z)\right] \nonumber \\
 \sim {} & \recip{2\pi}\left[\sqrt{\frac{2(\nu -1 - z)}{3z}}K_{1/3}\left(\frac{2^{3/2}}{3}\sqrt{\frac{(\nu - 1 - z)^3}{z}}\right) \right. \nonumber \\
  & \left. - \sqrt{\frac{2(\nu +1 - z)}{3z}}K_{1/3}\left(\frac{2^{3/2}}{3}\sqrt{\frac{(\nu + 1 - z)^3}{z}}\right)\right].
\end{align}
For our case
\begin{align}
\sqrt{\frac{\nu \pm 1 - z}{z}} = {} & (1 - e)^{1/2}\left[1 \pm \frac{(1-e)^{1/2}}{2\widetilde{f}} + \ldots \, \right];\\
\sqrt{\frac{(\nu \pm 1 - z)^{3/2}}{z}} = {} & \widetilde{f}\left[1 \pm \frac{3(1-e)^{1/2}}{2\widetilde{f}} + \ldots \, \right];
\end{align}
and so
\begin{align}
J'_n(ne) \sim {} & \recip{2\pi}\sqrt{\frac{2}{3}}(1-e)^{1/2}\left\{\left[1 - \frac{(1-e)^{1/2}}{2\widetilde{f}}\right]K_{1/3}\left(\frac{2^{3/2}\widetilde{f}}{3}\left[1 - \frac{3(1-e)^{1/2}}{2\widetilde{f}}\right]\right) \right. \nonumber \\
 & \left. - \left[1 + \frac{(1-e)^{1/2}}{2\widetilde{f}}\right]K_{1/3}\left(\frac{2^{3/2}\widetilde{f}}{3}\left[1 - \frac{3(1-e)^{1/2}}{2\widetilde{f}}\right]\right)\right\}\nonumber \\
 \sim {} & -\frac{1}{2\pi}\sqrt{\frac{2}{3}}(1-e)\left[2^{3/2}K'_{1/3}\left(\frac{2^{3/2}\widetilde{f}}{3}\right) + \recip{\widetilde{f}}K_{1/3}\left(\frac{2^{3/2}\widetilde{f}}{3}\right)\right].
\end{align}
We may re-express the derivative using the recurrence formula (Watson\cite{Watson1995} 3.71)
\begin{equation}
K_{\nu-1}(z) - K_{\nu+1}(z) = -2K'_\nu(z)
\end{equation}
to give
\begin{equation}
J'_n(ne) \sim \frac{1-e}{\sqrt{3}\pi}\left[K_{-2/3}\left(\frac{2^{3/2}\widetilde{f}}{3}\right) + K_{4/3}\left(\frac{2^{3/2}\widetilde{f}}{3}\right) - \recip{\sqrt{2}\widetilde{f}}K_{1/3}\left(\frac{2^{3/2}\widetilde{f}}{3}\right)\right].
\end{equation}
And so finally,
\begin{equation}
B(\widetilde{f}) = \recip{\sqrt{3}\pi}\left[K_{-2/3}\left(\frac{2^{3/2}\widetilde{f}}{3}\right) + K_{4/3}\left(\frac{2^{3/2}\widetilde{f}}{3}\right) - \recip{\sqrt{2}\widetilde{f}}K_{1/3}\left(\frac{2^{3/2}\widetilde{f}}{3}\right)\right],
\end{equation}
which is also well defined.

Having obtained expressions for $A(\widetilde{f})$ and $B(\widetilde{f})$ in terms of standard functions, we can now calculate the energy spectrum for a parabolic orbit. From \eqnref{PM_spectrum}
\begin{equation}
\diff{E}{f} = \frac{4\pi^2}{5}\frac{G^3}{c^5}\frac{M_\bullet^2\mu^2}{r\sub{p}^2}\ell\left(\frac{f}{f\sub{c}}\right),
\label{eq:PM_dEdf}
\end{equation}
where we have used the limit
\begin{align}
\ell(\widetilde{f}) = {} & \lim_{e \rightarrow 1}\left\{\ell(n,e)\right\} \nonumber \\
 = {} & \left[8\widetilde{f}B(\widetilde{f}) - 2\widetilde{f}A(\widetilde{f})\right]^2 + \left(128\widetilde{f}^4 + \frac{4\widetilde{f}^2}{3}\right)\left[A(\widetilde{f})\right]^2.
\end{align}

\subsection{Total Energy}

To check the validity of this limit we can calculate the total energy radiated. We should be able to calculate this by integrating \eqnref{PM_dEdf} over all frequencies, or by summing the energy radiated into each harmonic. For consistency, the two approaches should yield the same result. First, summing over harmonics
\begin{align}
E\sub{sum} = {} & \sum_n E(n) \nonumber \\
 = {} & \frac{64\pi}{5}\frac{G^3}{c^5}\frac{M_1^2M_2^2}{r\sub{p}^2}\omega\sub{c}(1-e)^{7/2}\sum_n g(n,e),
\end{align}
where we have used equations \eqref{eq:PM_P}, \eqref{eq:Kepler_freq} and \eqref{eq:E(n)}. Peters and Matthews\cite{Peters1963} provide the result
\begin{equation}
\sum_n g(n,e) = \left(1 + \frac{73}{24}e^2 + \frac{37}{96}e^4\right)(1-e^2)^{-7/2}.
\end{equation}
Using this,
\begin{equation}
E\sub{sum} = \frac{64\pi}{5}\frac{G^3}{c^5}\frac{M_1^2M_2^2}{r\sub{p}^2}\omega\sub{c}\left(1 + \frac{73}{24}e^2 + \frac{37}{96}e^4\right)(1-e^2)^{-7/2},
\end{equation}
which is perfectly well behaved as $e \rightarrow 1$. Taking the limit for a parabolic orbit, the total energy radiated is
\begin{equation}
E\sub{sum} = \frac{85\pi}{2^{5/2}3}\frac{G^3}{c^5}\frac{M_1^2M_2^2}{r\sub{p}^2}\omega\sub{c}.
\end{equation}
Integrating over the energy spectrum, \eqnref{PM_dEdf}, gives
\begin{align}
E\sub{int} = {} & \intd{0}{\infty}{\diff{E}{f}}{f} \nonumber \\
 = {} & \frac{2\pi}{5}\frac{G^3}{c^5}\frac{M_1^2M_2^2}{r\sub{p}^2}\omega\sub{c}\intd{0}{\infty}{\ell(\widetilde{f})}{\widetilde{f}}.
\end{align}
The integral can be evaluated numerically as
\begin{align}
\intd{0}{\infty}{\ell(\widetilde{f})}{\widetilde{f}} = {} & 12.5216858\ldots \nonumber \\
 = {} & \frac{425}{2^{7/2}3};
\end{align}
the two total energies are consistent
\begin{align}
\label{eq:PM_total}
E\sub{int} = {} & \frac{85\pi}{2^{5/2}3}\frac{G^3}{c^5}\frac{M_1^2M_2^2}{r\sub{p}^2}\omega\sub{c} \\
 = {} & E\sub{sum}.
\end{align}

\section{Applicability\label{sec:application}}

\subsection{Limit Of Approximation}

The approach of Peters and Matthews assumes Keplerian orbits in flat spacetime. This should be a valid approximation in the weak-field regime far from a massive body. To check the limit of this approximation, we may compare the PM approach with results from more accurate formalisms. Energy spectra for parabolic orbits do not seem to be available in the literature yet, so we will make do with the total energy fluxes calculated by Martel, who uses time-domain black hole perturbation theory\cite{Martel2004a}. \Figref{E_ratio} shows the ratio of the two energies as a function of periapsis distance.
\begin{figure}
% \includegraphics{}%
\caption{Ratio of the total energy radiated as calculated using the Peters and Matthews\cite{Peters1963} approach to that calculated by Martel\cite{Martel2004a} using black hole perturbation theory as a function of periapsis radius $r\sub{p}$. The latter approach should give more accurate results.\label{fig:E_ratio}}
\end{figure}
As expected the PM result is more accurate for larger periapses. The agreement worsens as the periapsis decreases. At $r\sub{p} = 4 M$, corresponding to the radius of the innermost stable circular orbit (ISCO), the energy flux calculated by Martel diverges, so the ratio tends to zero. This divergence is because in Schwarzschild (or Kerr) spacetime a parabolic orbit may have a zoom-whirl structure where it undergoes a number of near circular orbits (whirls) about the black hole. As the radius of the ISCO is approached, the number of whirls tends to infinity (in the absence of radiation reaction), so an infinite amount of energy is radiated. \Figref{N_ratio} shows how ratio of energies correlates with the number of orbits, defined as
\begin{equation}
N = \frac{\Delta \phi}{2\pi}
\end{equation}
\begin{figure}
% \includegraphics{}%
\caption{Ratio of the total energy radiated as calculated using the Peters and Matthews\cite{Peters1963} approach to that calculated by perturbation theory\cite{Martel2004a} as a function of the reciprocal of the number of orbits $1/N$. The Keplerian limit corresponds to $N = 1$.\label{fig:N_ratio}}
\end{figure}
As $N$ increases the ratio decreases as the Keplerian orbit does not take into account this extra rotation. The accuracy of the PM result deteriorates rapidly once the orbit transitions to a zoom-whirl trajectory. Once this happens the orbit is far from parabolic in shape.

The PM result is accurate to $\sim \SI{10}{\percent}$ for orbits with $N \lessim 1.1$. We will adopt this as a cut-off point. For an equatorial orbit in Kerr spacetime, the number of orbits is given by
\begin{align}
N = {} & \recip{\pi}\intd{r\sub{p}}{\infty}{\diff{\phi}{r}}{r} \nonumber
 = {} & \frac{L_z}{\pi\sqrt{2M}}\intd{r\sub{p}}{\infty}{\frac{r^2 - 2M(1 - a/L_z)r}{(r^2 - 2Mr + a^2)\sqrt{r^3 - (L_z^2/2M)r^2 + (L_z - a)^2r}}}{r},
\end{align}
where $L_z$ is the angular momentum about the $z$-axis, $a$ is the spin parameter and we have adopted units with $G = c = 1$. We will find it useful to define
\begin{equation}
r_\pm = M \pm \sqrt(M^2 - a^2},
\end{equation}
and the two non-zero roots of the cubic in the square root
\begin{equation}
r\sub{p, 1} = \frac{L_z^2}{2M} \pm \sqrt{\frac{L_z^4}{16M^2} - (L_z -a)^2},
\end{equation}
where the periapsis is the larger root $r\sub{p} > r_1$. The integral may be rewritten as
\begin{equation}
\frac{L_z}{\pi\sqrt{2M}}\intd{r\sub{p}}{\infty}{\recip{w}\left[1 + \frac{\alpha_+}{r-r_+} + \frac{\alpha_-}{r-r_-}\right]}{r},
\end{equation}
where
\begin{align}
w^2 = (r-r\sub{p})(r-r_1)r;
\alpha_\pm = {} & \pm\frac{2Mar_\pm - a^2L_z}{2L_z\sqrt{M^2-a^2}}.
\end{align}
This may be evaluated using elliptic integrals as\cite{Gradshteyn2000}
\begin{equation}
N = \frac{L}{\pi}\sqrt{\frac{2}{r\sub{p}M(M^2-a^2)}}\left[\left(\frac{Ma}{L_z} - {a^2}{2r_+}\right)\Pi\left(\frac{r_+}{r\sub{p}}\middle|\frac{r_1}{r\sub{p}}\right) - \left(\frac{Ma}{L_z} - {a^2}{2r_-}\right)\Pi\left(\frac{r_-}{r\sub{p}}\middle|\frac{r_1}{r\sub{p}}\right)\right],
\end{equation}
where $\Pi(n|m) = \intd{0}{pi/2}{(1-n\sin^2\theta)^{-1}(1-m\sin^2\theta)^{-1/2}}{\theta}$ is the complete elliptic integral of the third kind. In the limit of $a \rightarrow 0$ we recover the Schwarzschild result\cite{Cutler1994}
\begin{equation}
N = \frac{L}{\pi}\sqrt{\frac{2}{r\sub{p}M}}K\left(\frac{r_1}{r\sub{p}}\right),
\end{equation}
where $K(m) = \intd{0}{pi/2}{(1-m\sin^2\theta)^{-1/2}}{\theta}$ is the complete elliptic integral of the first kind. \Figref{N_ratio} shows the periapsis for which $N = 1.1$ for a range of spins.
\begin{figure}
% \includegraphics{}%
\caption{Periapsis radius corresponding to $N = 1.1$ as a function of spin parameter $a$.\label{fig:N_peri}}
\end{figure}
Equatorial orbits with larger periapses should be reasonably approximated by the PM result.

Non-equatorial orbits are more complicated because of the additional motion in the $\theta$ direction. This extra rotation will also mean that the PM approach is less accurate for non-equatorial orbits, and consequently a larger cut-off periapsis must be used.

\subsection{Astrophysical Implications}

Orbits with periapses of $r\sub{p} \lesssim 120 M$ should be detectable with LISA\cite{Rubbo2006, Hopman2007}. While the most interesting orbits --- those which probe the strong-field region of the MBH's spacetime --- are beyond the regime of the Peters and Matthews approach, $r\sub{p} \lesssim 20 M$ for equatorial orbits, we see that many will be reasonably approximated by the PM formalism. Therefore, it should be possible to explore this region of parameter space using the PM approximation as a guide.

% If in two-column mode, this environment will change to single-column
% format so that long equations can be displayed. Use
% sparingly.
%\begin{widetext}
% put long equation here
%\end{widetext}

% If you have acknowledgments, this puts in the proper section head.
\begin{acknowledgments}
CPLB is supported by an STFC studentship.
\end{acknowledgments}

% Create the reference section using BibTeX:
\bibliography{basename of .bib file}

\end{document}
